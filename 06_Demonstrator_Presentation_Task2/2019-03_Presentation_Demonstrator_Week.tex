% !TeX document-id = {6ded82bc-a22c-4503-a090-6b5ed60b2b6b}
% !TeX program = lualatex
% !TeX TXS-program:bibliography = txs:///bibtex

\documentclass[
xcolor=dvipsnames,
aspectratio=169,
9pt,
%handout,
]{beamer}

\definecolor{uniSblue}{RGB}{0,65,145}
\definecolor{uniSlightblue}{RGB}{0,190,255}
\definecolor{uniSgrey}{RGB}{62, 68, 76}
\definecolor{uniSyellow}{RGB}{255, 213, 0}

% colors not covered by CD but still useful
\definecolor{uniSred}{RGB}{230, 0, 50}
\definecolor{uniSgreen}{RGB}{0, 200, 50} 

%\usetheme{Luebeck}
\usetheme{boxes}
%\usecolortheme{beaver} %rot
\usecolortheme{seagull} %grau

%\usepackage[utf8]{inputenc}
\usepackage[T1]{fontenc}

\usepackage{fontspec}
\defaultfontfeatures{Ligatures=TeX}
%\usepackage[]{unicode-math} 
%\unimathsetup{math-style=TeX}
%\setmainfont[SmallCapsFeatures={LetterSpace=6}, Numbers={Proportional,OldStyle}, Ligatures=TeX]{Univers for UniS 55 Roman Rg}
%\setmonofont[Scale=0.9]{Hack}
%\setmainfont[SmallCapsFeatures={LetterSpace=6}, Numbers={Proportional,OldStyle}]{Minion Pro}
%\setmainfont[LetterSpace=3, Numbers={Proportional,OldStyle}, Ligatures=TeX]{Myriad Pro}
\defaultfontfeatures{Mapping=tex-text}
%

\usepackage{tikz, pgfplots}

\usepackage[ngerman, english]{babel}% Silbentrennung

%do not ovelap with footer line
\addtobeamertemplate{footnote}{}{\vspace{2.5ex}}

%footnotes without numbering with \blankfootnote{}
\let\svthefootnote\thefootnote
\textheight 1in
\newcommand\blankfootnote[1]{%
	\let\thefootnote\relax\footnotetext{\tiny #1}%
	\let\thefootnote\svthefootnote%
}



%item color setzen
\setbeamercolor{itemize item}{fg=uniSlightblue}
\setbeamercolor{itemize subitem}{fg=uniSlightblue}


\setbeamercolor{block title}{bg=uniSlightblue,fg=white}
\setbeamercolor{block body}{bg=uniSlightblue!10,fg=black}

\setbeamercolor*{structure}{bg=uniSlightblue,fg=uniSlightblue}

\setbeamercolor*{palette primary}{use=structure,fg=uniSlightblue,bg=structure.fg}
\setbeamercolor*{palette secondary}{use=structure,fg=uniSlightblue,bg=structure.fg!75}
\setbeamercolor*{palette tertiary}{use=structure,fg=white,bg=black}
\setbeamercolor*{palette quaternary}{fg=white,bg=black}

\setbeamercolor{section in toc}{fg=black,bg=white}
\setbeamercolor{alerted text}{use=structure,fg=structure.fg!50!black!80!black}

\setbeamercolor{titlelike}{parent=palette primary,fg=structure.fg!50!black}
\setbeamercolor{frametitle}{bg=white,fg=uniSgrey}

\setbeamercolor*{titlelike}{parent=palette primary}


%\setbeamercovered{transparent}

\setbeamertemplate{navigation symbols}{}%remove navigation symbols
\setbeamertemplate{itemize items}[circle]



% http://tex.stackexchange.com/questions/55806/tikzpicture-in-beamer/55827#55827
  \tikzset{
    invisible/.style={opacity=0},
    visible on/.style={alt=#1{}{invisible}},
    alt/.code args={<#1>#2#3}{%
      \alt<#1>{\pgfkeysalso{#2}}{\pgfkeysalso{#3}} % \pgfkeysalso doesn't change the path
    },
  }
  \tikzset{onslide/.code args={<#1>#2}{%
  \only<#1>{\pgfkeysalso{#2}} % \pgfkeysalso doesn't change the path
}}




\newcommand{\partitle}[1]{\textcolor{uniSlightblue}{\textbf{#1}}\\ \smallskip}

% argument #1: any options
\newenvironment{customlegend}[1][]{%
	\begingroup
	% inits/clears the lists (which might be populated from previous
	% axes):
	\csname pgfplots@init@cleared@structures\endcsname
	\pgfplotsset{#1}%
}{%
	% draws the legend:
	\csname pgfplots@createlegend\endcsname
	\endgroup
}%

\pgfplotsset{compat=newest}


% makes \addlegendimage available (typically only available within an
% axis environment):
\def\addlegendimage{\csname pgfplots@addlegendimage\endcsname}
\def\addlegendentry{\csname pgfplots@addlegendentry\endcsname}




\usepackage{bm}

\usepackage{wasysym}

% possibly needed TikZ libraries
\usetikzlibrary{calc,fit, positioning, patterns, decorations.pathmorphing, decorations.markings, shapes, shapes.geometric, shapes.callouts, arrows.meta, shadings, trees, chains, 3d, shadows}

% Toleranzen mit tikz darstellen
%\input{tikzTolerances}

% camera shape. 
% Draw with \camera{(xshift, yshift)}{rotationDeg}{Label}
\def\camera#1#2#3{
	\begin{scope}[shift={#1}, rotate=#2, fill=white]
		\draw [fill=black](0,0) -- (0.5,0.75) -- (-0.5,0.75) -- cycle;
		\node [fill=white,thick, minimum height=1cm, draw] at (0,0){#3};
	\end{scope}
}

% TikZ blocks

% mechanical:
\tikzstyle{spring}=[thick,decorate,decoration={zigzag,pre length=0.3cm,post length=0.3cm,segment length=6}]
\tikzstyle{damper}=[thick,decoration={markings,  
	mark connection node=dmp,
	mark=at position 0.5 with 
	{
		\node (dmp) [thick,inner sep=0pt,transform shape,rotate=-90,minimum width=15pt,minimum height=3pt,draw=none] {};
		\draw [thick] ($(dmp.north east)+(2pt,0)$) -- (dmp.south east) -- (dmp.south west) -- ($(dmp.north west)+(2pt,0)$);
		\draw [thick] ($(dmp.north)+(0,-5pt)$) -- ($(dmp.north)+(0,5pt)$);
	}
}, decorate]
\tikzstyle{ground}=[fill,pattern=north east lines,draw=none,minimum width=0.75cm,minimum height=0.3cm]

% electrical and control theory:
\tikzstyle{block} = [draw=black, fill=white, rectangle, text centered, minimum height=3em, minimum width=3em]
\tikzstyle{sum} = [draw, circle, node distance=1cm]
\tikzstyle{input} = [coordinate]
\tikzstyle{output} = [coordinate]
\tikzstyle{new} = [fill=uniSlightblue!20]
\tikzstyle{gain} = [regular polygon, regular polygon sides=3,
draw, fill=white, text width=1em,
inner sep=1mm, outer sep=0mm,
shape border rotate=-90]

\tikzset{cross/.style={cross out, draw, minimum size=2*(#1-\pgflinewidth), inner sep=0pt, outer sep=0pt, line width=4pt}}

% more conveient arrow heads
\tikzset{>={Latex[width=2mm,length=2mm]}}

% line width global definition for unique layout
\tikzset{every picture/.style={line width=0.6pt}}


% Fancy node streching from http://tex.stackexchange.com/a/124507
% Examples:
%
%\node[right=of |(A)(B)(D)]          {Z};
%
%\node[left=of |(A)(B), ultra thick] {Y};
%\node[left=of |(C)(D), minimum width=50pt]          (X) {X};
%
%\node[span vertical=(A)(B), above=of -X] {y};
%
%\node[below=of -(A)(B)(C)(D)] {0};
\makeatletter
\def\pgfutil@firstofmany#1#2\pgf@stop{#1}
\def\pgfutil@secondofmany#1#2\pgf@stop{#2}
\def\tikz@lib@place@of@#1#2#3{%
	\def\pgf@tempa{fit bounding box}%
	\edef\pgf@temp{\expandafter\pgfutil@firstofmany#2\pgf@stop}
	\if\pgf@temp(%
	\tikz@lib@place@fit@scan{#2}{0}%
	\else\if\pgf@temp|
	\expandafter\tikz@lib@place@fit@scan\expandafter{\pgfutil@secondofmany#2\pgf@stop}{1}%
	\else\ifx\pgf@temp\tikz@activebar
	\expandafter\tikz@lib@place@fit@scan\expandafter{\pgfutil@secondofmany#2\pgf@stop}{1}%
	\else\if\pgf@temp-
	\expandafter\tikz@lib@place@fit@scan\expandafter{\pgfutil@secondofmany#2\pgf@stop}{2}%
	\else\if\pgf@temp+
	\expandafter\tikz@lib@place@fit@scan\expandafter{\pgfutil@secondofmany#2\pgf@stop}{3}%
	\else
	\def\pgf@tempa{#2}%
	\fi
	\fi
	\fi
	\fi
	\fi
	\expandafter\tikz@scan@one@point\expandafter\tikz@lib@place@remember\expandafter(\pgf@tempa)%
	\iftikz@shapeborder%
	% Ok, this is relative to a border.
	\iftikz@lib@ignore@size%
	\edef\tikz@node@at{\noexpand\pgfpointanchor{\tikz@shapeborder@name}{center}}%
	\def\tikz@anchor{center}%
	\else%
	\edef\tikz@node@at{\noexpand\pgfpointanchor{\tikz@shapeborder@name}{#3}}%
	\fi%
	\fi%
	\edef\tikz@lib@place@nums{#1}%
}
\def\tikz@lib@place@fit@scan#1#2{
	\pgf@xb=-16000pt\relax%
	\pgf@xa=16000pt\relax%
	\pgf@yb=-16000pt\relax%
	\pgf@ya=16000pt\relax%
	\if\pgfutil@firstofmany#1\pgf@stop(%
	\tikz@lib@fit@scan#1\pgf@stop%
	\else
	\tikz@lib@fit@scan(#1)\pgf@stop
	\fi
	\ifdim\pgf@xa>\pgf@xa
	% shouldn't happen
	\else
	\expandafter\def\csname pgf@sh@ns@fit bounding box\endcsname{rectangle}%
	\expandafter\edef\csname pgf@sh@np@fit bounding box\endcsname{%
		\def\noexpand\southwest{\noexpand\pgfqpoint{\the\pgf@xa}{\the\pgf@ya}}%
		\def\noexpand\northeast{\noexpand\pgfqpoint{\the\pgf@xb}{\the\pgf@yb}}%
	}%
	\expandafter\def\csname pgf@sh@nt@fit bounding box\endcsname{{1}{0}{0}{1}{0pt}{0pt}}%
	\expandafter\def\csname pgf@sh@pi@fit bounding box\endcsname{\pgfpictureid}%
	\ifcase#2\relax
	\or % 1 = vertical
	\pgf@y=\pgf@yb%
	\advance\pgf@y by-\pgf@ya%
	\edef\pgf@marshal{\noexpand\tikzset{minimum height={\the\pgf@y-2*(\noexpand\pgfkeysvalueof{/pgf/outer ysep})}}}%
	\pgf@marshal
	\or % 2 = horizontal
	\pgf@x=\pgf@xb%
	\advance\pgf@x by-\pgf@xa%
	\edef\pgf@marshal{\noexpand\tikzset{minimum width={\the\pgf@x-2*(\noexpand\pgfkeysvalueof{/pgf/outer xsep})}}}%
	\pgf@marshal
	\or % 3 = both directions
	\pgf@y=\pgf@yb%
	\advance\pgf@y by-\pgf@ya%
	\pgf@x=\pgf@xb%
	\advance\pgf@x by-\pgf@xa%
	\edef\pgf@marshal{\noexpand\tikzset{minimum height={\the\pgf@y-2*(\noexpand\pgfkeysvalueof{/pgf/outer ysep})},minimum width={\the\pgf@x-2*(\noexpand\pgfkeysvalueof{/pgf/outer xsep})}}}%
	\pgf@marshal
	\fi
	\fi
}
\tikzset{
	fit bounding box/.code={\tikz@lib@place@fit@scan{#1}{0}},
	span vertical/.code={\tikz@lib@place@fit@scan{#1}{1}},
	span horizontal/.code={\tikz@lib@place@fit@scan{#1}{2}},
	span/.code={\tikz@lib@place@fit@scan{#1}{3}}}

\makeatother


%pgfplots settings
\pgfplotsset{
	compat = newest,
	%tick label style = {font=\sansmath\sffamily},
	%every axis label = {font=\sansmath\sffamily},
	%legend style = {font=\sansmath\sffamily},
	%label style = {font=\sansmath\sffamily},
	grid=major,
	every axis plot/.append style={thick},
}

%% 3D drawings

% 3D drawings axes mapping onto 2D image projection
\tikzset{math3d/.style= {x={(1cm, 0cm)}, y={(.353cm,.353cm)}, z={(0cm,1cm)}}}

% other 3D drawing views
\tikzstyle{isometric}=[x={(0.710cm,-0.410cm)},y={(0cm,0.820cm)},z={(-0.710cm,-0.410cm)}]
\tikzstyle{dimetric} =[x={(0.935cm,-0.118cm)},y={(0cm,0.943cm)},z={(-0.354cm,-0.312cm)}]
\tikzstyle{dimetric2}=[x={(0.935cm,-0.118cm)},z={(0cm,0.943cm)},y={(+0.354cm,+0.312cm)}]
\tikzstyle{dimetric3} =[y={(0.935cm,-0.118cm)},z={(0cm,0.943cm)},x={(-0.354cm,-0.312cm)}]
\tikzstyle{trimetric}=[x={(0.926cm,-0.207cm)},y={(0cm,0.837cm)},z={(-0.378cm,-0.507cm)}]

%Call: \tikzblock{Breite}{Höhe}{Tiefe}{Zusätze}
\newcommand{\tikzblock}[4]{\filldraw[#4] (0,0,0) rectangle (#1,0,#2);
	\filldraw[fill=black!5, #4] (0,0,#2)--(0,#3,#2) --(#1,#3,#2) --(#1,0,#2) -- cycle;
	\filldraw[fill=black!20, #4] (#1,0,0) -- (#1,0,#2)-- (#1,#3,#2) -- (#1,#3,0) -- cycle;}

\newcommand{\colouredtikzblock}[4]{\filldraw[fill=uniSlightblue!50, draw=uniSlightblue!50, #4] (0,0,0) rectangle (#1,0,#2);
	\filldraw[fill=uniSlightblue!30, draw=uniSlightblue!50,#4] (0,0,#2)--(0,#3,#2) --(#1,#3,#2) --(#1,0,#2) -- cycle;
	\filldraw[fill=uniSlightblue, draw=uniSlightblue!50,#4] (#1,0,0) -- (#1,0,#2)-- (#1,#3,#2) -- (#1,#3,0) -- cycle;}

%Aufruf: \tikzplaneXZ{x-Dimension}{z-Dimension}{Zusätze}
\newcommand{\tikzplaneXZ}[3]{\filldraw[#3] (0,0,0) -- (#1,0,0) -- (#1,0,#2) -- (0,0,#2) -- cycle;}

%Aufruf: \tikzplaneXZ{x-Dimension}{y-Dimension}{Zusätze}
\newcommand{\tikzplaneXY}[3]{\filldraw[fill=black!5,#3] (0,0,0) -- (#1,0,0) -- (#1,#2,0) -- (0,#2,0) -- cycle;}

%Aufruf: \tikzplaneYZ{y-Dimension}{z-Dimension}{Zusätze}
\newcommand{\tikzplaneYZ}[3]{\filldraw[fill=black!20,#3] (0,0,0) -- (0,#1,0) -- (0,#1,#2) -- (0,0,#2) -- cycle;}

% Pandocs Tightlist command
\providecommand{\tightlist}{%
	\setlength{\itemsep}{0pt}\setlength{\parskip}{0pt}}



%external TikZ, pgf for faster compiling and exportable figures
%\usetikzlibrary{external}
%\tikzset{external/mode=graphics if exists}
%\tikzexternalize[prefix=..img/tikz/]
%%\tikzset{external/force remake}

%matlab2tikz options
\newlength\fheight 
\newlength\fwidth 
\setlength\fheight{3cm} 
\setlength\fwidth{12cm}

\newcommand{\fillhalfrb}[3]{%
	\begin{scope}[on background layer]
		\fill[fill=#2] (#1.north west)--(#1.south west)--(#1.north east)--cycle;
		\fill[fill=#3] (#1.north east)--(#1.south east)--(#1.south west)--cycle;
	\end{scope}	
}

%Strechable skip to get contents a the bottom of slides
\newcommand{\btVFill}{\vskip0pt plus 1filll}

\makeatletter
\newcommand*{\defeq}{\mathrel{\rlap{%
			\raisebox{0.3ex}{$\m@th\cdot$}}%
		\raisebox{-0.3ex}{$\m@th\cdot$}}%
	=}
\makeatother

\newcommand{\ddiff}{\mathrm{d}}
\newcommand{\R}{\ensuremath{\mathbb{R}}}
\newcommand{\C}{\ensuremath{\mathbb{C}}}

\newcommand{\bmat}[1]{\ensuremath{\begin{bmatrix} #1 \end{bmatrix}}}

\newcommand{\imgsource}[1]{\begin{flushright}
		\textcolor{black!40}{\tiny Source:~#1}	
\end{flushright}}

\newcommand{\follows}{\ensuremath{\Rightarrow}}
\newcommand{\gdw}{\ensuremath{\Leftrightarrow}}

\usetikzlibrary{backgrounds}

%\usepackage{tikz-uml}
\usepackage{multimedia}
\usepackage{media9} 
\graphicspath{{./img/}}

%%TITLE
\defbeamertemplate*{title page}{customized}[1][]
{
	\begin{tikzpicture}[remember picture,overlay]
		\filldraw[uniSgrey!20]([yshift=-40pt]current page.north west) rectangle (current page.south east);
		\node[anchor=north west] 
		at ([yshift=-4pt, xshift=10pt]current page.north west) (logo)
		{\parbox[t]{.19\paperwidth}{\raggedleft%
				\usebeamercolor[fg]{titlegraphic}\inserttitlegraphic}};
		
		\node[anchor= west] 
		at ([xshift=4em, yshift=-20pt]current page.west) (STRimage)
		{\includegraphics[width=.4\paperwidth]{../img/STR}};
		
		\node[fill=uniSgrey!80!black, circle, minimum size=18em, inner sep=0.3em, align=left, text width=14em] at ([xshift=-5cm, yshift=-0.5cm]current page.east) {
		\begin{minipage}[][5cm][c]{5cm}	
			{\raggedleft\scriptsize \textcolor{white}{\insertinstitute}}\\\vfill
			{\raggedleft\usebeamerfont{title}\textcolor{white}{\normalsize\textbf{\inserttitle}}}\\
			{\raggedleft\usebeamerfont{subtitle}\textcolor{white}{\small\textit{\insertsubtitle}}}\\\vfill
			{\raggedleft\usebeamerfont{date}\textcolor{white}{\footnotesize\insertdate}}
		\end{minipage}
		};
		\node[fill=white, circle, minimum size=2em, inner sep=5pt, align=center, text=uniSgrey, font=\footnotesize\selectfont] at ([xshift=-6em, yshift=-8em]current page.east) {\authorForTitle};
	\end{tikzpicture}	
}

%%TITLE_END



\title[Design and implementation of a teleoperated drawing robot]{Design and implementation of a teleoperated drawing robot}


\author{F. Jaensch, J. Terfurth, M. Wnuk, Z. Chen, D. Tomzik, J. Shahabi, A. Garrett, Kandasamy S., M. Mahmoudinezhad}
\newcommand{\authorForTitle}{F. Jaensch, J. Terfurth,\\ M. Wnuk, Z. Chen,\\ D. Tomzik, J. Shahabi,\\ A. Garrett, Kandasamy S., \\ M. Mahmoudinezhad}
\titlegraphic{\includegraphics[height=30pt]{LogosUSTUTT_UOA}}
\institute{DFG IRTG GRK2198/1\newline Soft Tissue Robotics}
\date{26 March 2019}


% This is only inserted into the PDF information catalog. Can be left
% out.
\subject{IRTG Project presentation}

\begin{document}
	
	\addtobeamertemplate{frametitle}{}{%
		\begin{tikzpicture}[remember picture,overlay]
		\node[anchor=north east,yshift=3pt,xshift=0pt] at (current page.north east)
		{\includegraphics[height=0.8cm]{img/LogosUSTUTT_UOA}};
		\end{tikzpicture}}
	
	%{\usebackgroundtemplate{\includegraphics[height=1.0\paperheight]{img/cnc_milling_low.jpg}}
		\begin{frame}
		\titlepage
		\end{frame}%}

\setbeamertemplate{footline}[text line]{%
	\parbox{\linewidth}{\vspace*{-8pt}\insertframenumber\hfill %\insertshortauthor:~
    \insertshorttitle\hfill}}
\setbeamertemplate{navigation symbols}{}

\AtBeginSection[]{
\begin{frame}
\frametitle{Contents}
\tableofcontents[ 
currentsubsection, 
hideothersubsections, 
sectionstyle=show/shaded, 
subsectionstyle=show/shaded, 
] 
%You might wish to add the option [pausesections]
\end{frame}
}

\AtBeginSubsection[]{
	\begin{frame}
	\frametitle{Contents}
	\tableofcontents[ 
	currentsubsection, 
	hideothersubsections, 
	sectionstyle=show/hide, 
	subsectionstyle=show/shaded, 
	] 
	%You might wish to add the option [pausesections]
	\end{frame}
}


%% Build a slide like this and place LaTeX content within
%\begin{frame}{Frame title}
%
%\end{frame}
% ----------------------------------------------------------------------------------------------------------
% ----------------------------------------------------------------------------------------------------------
%\section{Motivation}
%% ----------------------------------------------------------------------------------------------------------
%
%\begin{frame}{What is intelligence?}
%	\begin{tabular}{c p{10cm}}
%		\structure{Definition:} & "...the resultant of the process of \structure{acquiring}, \structure{storing} in memory, \structure{retrieving}, \structure{combining}, \structure{comparing}, and \structure{using} in new contexts \structure{information} and conceptual skills" \\& 		- Lloyd Humphreys
%	\end{tabular}
%	
%		\begin{center}
%		\movie[width=0.5\linewidth,
%		height=0.28125\linewidth,poster,loop]{}{vid/MotivationVid_Challenge_Ape.mov}
%	\end{center}	
%	
%%	\begin{center}	
%%		\includemedia[
%%		height=4.5cm,
%%		keepaspectratio,
%%		activate=pageopen,
%%		addresource=vid/MotivationVid_Challenge_final.MP4,
%%		flashvars={source=vid/MotivationVid_Challenge_final.MP4}
%%		]{}{VPlayer.swf}
%%	\end{center} 
%\end{frame}

% ----------------------------------------------------------------------------------------------------------
%\begin{frame}{Industrial Challenges \& Applications}
%	
%	\begin{center}
%		\movie[width=0.8\linewidth,
%		height=0.45\linewidth,poster]{}{vid/MotivationVid_Industrial.mov}
%	\end{center}
%	
%	
%%	\begin{center}
%%		\includemedia[
%%			width=0.5\linewidth,
%%			height=0.45\linewidth,
%%			activate=pageopen,
%%			addresource=vid/MotivationVid_Industrial.MP4,
%%			flashvars={source=vid/MotivationVid_Industrial.MP4}
%%			]{}{VPlayer.swf}
%%		\end{center} 	
%	
%\end{frame}
%
%\begin{frame}{Medical Challenges \& Applications}
%	\begin{center}
%		\movie[width=0.8\linewidth,
%		height=0.45\linewidth,poster]{}{vid/MotivationVid_Medical.mov}
%	\end{center}	
%	
%%	\includemedia[
%%		width=0.8\linewidth,
%%		height=0.45\linewidth,
%%		activate=pageopen,
%%		addresource=vid/MotivationVid_Medical.MP4,
%%		flashvars={source=vid/MotivationVid_Medical.MP4}
%%		]{}{VPlayer.swf}
%\end{frame}



% ----------------------------------------------------------------------------------------------------------
\section{Setup description}
\begin{frame}{Motivation}
	\begin{columns}
		\begin{column}{0.4\textwidth}
			\begin{itemize}
				\item Say why our task is important
				\item ...
			\end{itemize}
		\end{column}
		\begin{column}{0.6\textwidth}
			\begin{center}
				\begin{figure}
					\includegraphics[width=1\textwidth]{img/Wolowitz.jpg}
					\caption{Robotic teleoperation in an everyday scenario. Picture taken from the SitCom ``The Big Bang Theory''}
				\end{figure}
			\end{center}
		\end{column}
	\end{columns}
\end{frame}

\begin{frame}{Goal}
	\begin{columns}
		\begin{column}{0.4\textwidth}
			\begin{itemize}
				\item Define goal set for the demonstrator week here
				\item ...
			\end{itemize}
		\end{column}
		\begin{column}{0.6\textwidth}
			\begin{center}
				\begin{figure}
					\resizebox{0.6\columnwidth}{!}{
					\includegraphics[width=1\textwidth]{img/ConceptDrawing.jpg}
					}
					\caption{Original concept for the 2019 summer school demonstrator}
				\end{figure}
			\end{center}
		\end{column}
	\end{columns}
\end{frame}

\begin{frame}{Problem formulation}
	\begin{columns}
		\begin{column}{0.4\textwidth}
			\begin{itemize}
				\item Here goes the description of the involved problems
				\item Force feedback from soft tissue
				\item control for the system
				\item sensor
				\item communication and interfaces
				\item ...
			\end{itemize}
		\end{column}
		\begin{column}{0.6\textwidth}
			\begin{center}
				\begin{figure}
					\resizebox{0.8\columnwidth}{!}{
						\includegraphics[width=1\textwidth]{img/ProblemDescription_BallandPen.jpg}
					}
					\caption{Balloon and pen during a drawing task. The pointy tip of the pen threatens to burst the balloon.}
				\end{figure}
			\end{center}
		\end{column}
	\end{columns}
\end{frame}


%-------------------------------------------------------------------------------

\section{Concept and realization}

\begin{frame}{Concept of the teleoperated drawing robot}
	
	Figure of the overall concept ...
	
\end{frame}


\begin{frame}{Sensor development}

	Explain how the sensors were developed and their working principle
	
\end{frame}

% overview slide:


\begin{frame}{Communication}

	Explain the communication with MQTT Broker - IoT Style

\end{frame}

\begin{frame}{Control}

	Explain the control logic and how the robot is teleoperated

\end{frame}


\section{Results}

\begin{frame}{Teleoperated Robot motion}
	Show video of the teleoperated robot motion here ...
\end{frame}

\begin{frame}{Force Feedback}
	Show video of the force feedback to the haptic device here ...
\end{frame}

\begin{frame}{Drawing}
	Show video/results of teleoperated drawig ...
\end{frame} 

%---------------------------------------------------------------------------

\section{Conclusion}

\begin{frame}{Conclusion \& next steps}

	Come up with some conclusions here ...

\end{frame}


\begin{frame}{Thank you!}
	
	Say thanks to the PIs here ...
	
\end{frame}

\end{document}

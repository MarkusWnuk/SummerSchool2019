\usepackage{bm}

\usepackage{wasysym}

% possibly needed TikZ libraries
\usetikzlibrary{calc,fit, positioning, patterns, decorations.pathmorphing, decorations.markings, shapes, shapes.geometric, shapes.callouts, arrows.meta, shadings, trees, chains, 3d, shadows}

% Toleranzen mit tikz darstellen
%\input{tikzTolerances}

% camera shape. 
% Draw with \camera{(xshift, yshift)}{rotationDeg}{Label}
\def\camera#1#2#3{
	\begin{scope}[shift={#1}, rotate=#2, fill=white]
		\draw [fill=black](0,0) -- (0.5,0.75) -- (-0.5,0.75) -- cycle;
		\node [fill=white,thick, minimum height=1cm, draw] at (0,0){#3};
	\end{scope}
}

% TikZ blocks

% mechanical:
\tikzstyle{spring}=[thick,decorate,decoration={zigzag,pre length=0.3cm,post length=0.3cm,segment length=6}]
\tikzstyle{damper}=[thick,decoration={markings,  
	mark connection node=dmp,
	mark=at position 0.5 with 
	{
		\node (dmp) [thick,inner sep=0pt,transform shape,rotate=-90,minimum width=15pt,minimum height=3pt,draw=none] {};
		\draw [thick] ($(dmp.north east)+(2pt,0)$) -- (dmp.south east) -- (dmp.south west) -- ($(dmp.north west)+(2pt,0)$);
		\draw [thick] ($(dmp.north)+(0,-5pt)$) -- ($(dmp.north)+(0,5pt)$);
	}
}, decorate]
\tikzstyle{ground}=[fill,pattern=north east lines,draw=none,minimum width=0.75cm,minimum height=0.3cm]

% electrical and control theory:
\tikzstyle{block} = [draw=black, fill=white, rectangle, text centered, minimum height=3em, minimum width=3em]
\tikzstyle{sum} = [draw, circle, node distance=1cm]
\tikzstyle{input} = [coordinate]
\tikzstyle{output} = [coordinate]
\tikzstyle{new} = [fill=uniSlightblue!20]
\tikzstyle{gain} = [regular polygon, regular polygon sides=3,
draw, fill=white, text width=1em,
inner sep=1mm, outer sep=0mm,
shape border rotate=-90]

\tikzset{cross/.style={cross out, draw, minimum size=2*(#1-\pgflinewidth), inner sep=0pt, outer sep=0pt, line width=4pt}}

% more conveient arrow heads
\tikzset{>={Latex[width=2mm,length=2mm]}}

% line width global definition for unique layout
\tikzset{every picture/.style={line width=0.6pt}}


% Fancy node streching from http://tex.stackexchange.com/a/124507
% Examples:
%
%\node[right=of |(A)(B)(D)]          {Z};
%
%\node[left=of |(A)(B), ultra thick] {Y};
%\node[left=of |(C)(D), minimum width=50pt]          (X) {X};
%
%\node[span vertical=(A)(B), above=of -X] {y};
%
%\node[below=of -(A)(B)(C)(D)] {0};
\makeatletter
\def\pgfutil@firstofmany#1#2\pgf@stop{#1}
\def\pgfutil@secondofmany#1#2\pgf@stop{#2}
\def\tikz@lib@place@of@#1#2#3{%
	\def\pgf@tempa{fit bounding box}%
	\edef\pgf@temp{\expandafter\pgfutil@firstofmany#2\pgf@stop}
	\if\pgf@temp(%
	\tikz@lib@place@fit@scan{#2}{0}%
	\else\if\pgf@temp|
	\expandafter\tikz@lib@place@fit@scan\expandafter{\pgfutil@secondofmany#2\pgf@stop}{1}%
	\else\ifx\pgf@temp\tikz@activebar
	\expandafter\tikz@lib@place@fit@scan\expandafter{\pgfutil@secondofmany#2\pgf@stop}{1}%
	\else\if\pgf@temp-
	\expandafter\tikz@lib@place@fit@scan\expandafter{\pgfutil@secondofmany#2\pgf@stop}{2}%
	\else\if\pgf@temp+
	\expandafter\tikz@lib@place@fit@scan\expandafter{\pgfutil@secondofmany#2\pgf@stop}{3}%
	\else
	\def\pgf@tempa{#2}%
	\fi
	\fi
	\fi
	\fi
	\fi
	\expandafter\tikz@scan@one@point\expandafter\tikz@lib@place@remember\expandafter(\pgf@tempa)%
	\iftikz@shapeborder%
	% Ok, this is relative to a border.
	\iftikz@lib@ignore@size%
	\edef\tikz@node@at{\noexpand\pgfpointanchor{\tikz@shapeborder@name}{center}}%
	\def\tikz@anchor{center}%
	\else%
	\edef\tikz@node@at{\noexpand\pgfpointanchor{\tikz@shapeborder@name}{#3}}%
	\fi%
	\fi%
	\edef\tikz@lib@place@nums{#1}%
}
\def\tikz@lib@place@fit@scan#1#2{
	\pgf@xb=-16000pt\relax%
	\pgf@xa=16000pt\relax%
	\pgf@yb=-16000pt\relax%
	\pgf@ya=16000pt\relax%
	\if\pgfutil@firstofmany#1\pgf@stop(%
	\tikz@lib@fit@scan#1\pgf@stop%
	\else
	\tikz@lib@fit@scan(#1)\pgf@stop
	\fi
	\ifdim\pgf@xa>\pgf@xa
	% shouldn't happen
	\else
	\expandafter\def\csname pgf@sh@ns@fit bounding box\endcsname{rectangle}%
	\expandafter\edef\csname pgf@sh@np@fit bounding box\endcsname{%
		\def\noexpand\southwest{\noexpand\pgfqpoint{\the\pgf@xa}{\the\pgf@ya}}%
		\def\noexpand\northeast{\noexpand\pgfqpoint{\the\pgf@xb}{\the\pgf@yb}}%
	}%
	\expandafter\def\csname pgf@sh@nt@fit bounding box\endcsname{{1}{0}{0}{1}{0pt}{0pt}}%
	\expandafter\def\csname pgf@sh@pi@fit bounding box\endcsname{\pgfpictureid}%
	\ifcase#2\relax
	\or % 1 = vertical
	\pgf@y=\pgf@yb%
	\advance\pgf@y by-\pgf@ya%
	\edef\pgf@marshal{\noexpand\tikzset{minimum height={\the\pgf@y-2*(\noexpand\pgfkeysvalueof{/pgf/outer ysep})}}}%
	\pgf@marshal
	\or % 2 = horizontal
	\pgf@x=\pgf@xb%
	\advance\pgf@x by-\pgf@xa%
	\edef\pgf@marshal{\noexpand\tikzset{minimum width={\the\pgf@x-2*(\noexpand\pgfkeysvalueof{/pgf/outer xsep})}}}%
	\pgf@marshal
	\or % 3 = both directions
	\pgf@y=\pgf@yb%
	\advance\pgf@y by-\pgf@ya%
	\pgf@x=\pgf@xb%
	\advance\pgf@x by-\pgf@xa%
	\edef\pgf@marshal{\noexpand\tikzset{minimum height={\the\pgf@y-2*(\noexpand\pgfkeysvalueof{/pgf/outer ysep})},minimum width={\the\pgf@x-2*(\noexpand\pgfkeysvalueof{/pgf/outer xsep})}}}%
	\pgf@marshal
	\fi
	\fi
}
\tikzset{
	fit bounding box/.code={\tikz@lib@place@fit@scan{#1}{0}},
	span vertical/.code={\tikz@lib@place@fit@scan{#1}{1}},
	span horizontal/.code={\tikz@lib@place@fit@scan{#1}{2}},
	span/.code={\tikz@lib@place@fit@scan{#1}{3}}}

\makeatother


%pgfplots settings
\pgfplotsset{
	compat = newest,
	%tick label style = {font=\sansmath\sffamily},
	%every axis label = {font=\sansmath\sffamily},
	%legend style = {font=\sansmath\sffamily},
	%label style = {font=\sansmath\sffamily},
	grid=major,
	every axis plot/.append style={thick},
}

%% 3D drawings

% 3D drawings axes mapping onto 2D image projection
\tikzset{math3d/.style= {x={(1cm, 0cm)}, y={(.353cm,.353cm)}, z={(0cm,1cm)}}}

% other 3D drawing views
\tikzstyle{isometric}=[x={(0.710cm,-0.410cm)},y={(0cm,0.820cm)},z={(-0.710cm,-0.410cm)}]
\tikzstyle{dimetric} =[x={(0.935cm,-0.118cm)},y={(0cm,0.943cm)},z={(-0.354cm,-0.312cm)}]
\tikzstyle{dimetric2}=[x={(0.935cm,-0.118cm)},z={(0cm,0.943cm)},y={(+0.354cm,+0.312cm)}]
\tikzstyle{dimetric3} =[y={(0.935cm,-0.118cm)},z={(0cm,0.943cm)},x={(-0.354cm,-0.312cm)}]
\tikzstyle{trimetric}=[x={(0.926cm,-0.207cm)},y={(0cm,0.837cm)},z={(-0.378cm,-0.507cm)}]

%Call: \tikzblock{Breite}{Höhe}{Tiefe}{Zusätze}
\newcommand{\tikzblock}[4]{\filldraw[#4] (0,0,0) rectangle (#1,0,#2);
	\filldraw[fill=black!5, #4] (0,0,#2)--(0,#3,#2) --(#1,#3,#2) --(#1,0,#2) -- cycle;
	\filldraw[fill=black!20, #4] (#1,0,0) -- (#1,0,#2)-- (#1,#3,#2) -- (#1,#3,0) -- cycle;}

\newcommand{\colouredtikzblock}[4]{\filldraw[fill=uniSlightblue!50, draw=uniSlightblue!50, #4] (0,0,0) rectangle (#1,0,#2);
	\filldraw[fill=uniSlightblue!30, draw=uniSlightblue!50,#4] (0,0,#2)--(0,#3,#2) --(#1,#3,#2) --(#1,0,#2) -- cycle;
	\filldraw[fill=uniSlightblue, draw=uniSlightblue!50,#4] (#1,0,0) -- (#1,0,#2)-- (#1,#3,#2) -- (#1,#3,0) -- cycle;}

%Aufruf: \tikzplaneXZ{x-Dimension}{z-Dimension}{Zusätze}
\newcommand{\tikzplaneXZ}[3]{\filldraw[#3] (0,0,0) -- (#1,0,0) -- (#1,0,#2) -- (0,0,#2) -- cycle;}

%Aufruf: \tikzplaneXZ{x-Dimension}{y-Dimension}{Zusätze}
\newcommand{\tikzplaneXY}[3]{\filldraw[fill=black!5,#3] (0,0,0) -- (#1,0,0) -- (#1,#2,0) -- (0,#2,0) -- cycle;}

%Aufruf: \tikzplaneYZ{y-Dimension}{z-Dimension}{Zusätze}
\newcommand{\tikzplaneYZ}[3]{\filldraw[fill=black!20,#3] (0,0,0) -- (0,#1,0) -- (0,#1,#2) -- (0,0,#2) -- cycle;}

% Pandocs Tightlist command
\providecommand{\tightlist}{%
	\setlength{\itemsep}{0pt}\setlength{\parskip}{0pt}}



%external TikZ, pgf for faster compiling and exportable figures
%\usetikzlibrary{external}
%\tikzset{external/mode=graphics if exists}
%\tikzexternalize[prefix=..img/tikz/]
%%\tikzset{external/force remake}

%matlab2tikz options
\newlength\fheight 
\newlength\fwidth 
\setlength\fheight{3cm} 
\setlength\fwidth{12cm}

\newcommand{\fillhalfrb}[3]{%
	\begin{scope}[on background layer]
		\fill[fill=#2] (#1.north west)--(#1.south west)--(#1.north east)--cycle;
		\fill[fill=#3] (#1.north east)--(#1.south east)--(#1.south west)--cycle;
	\end{scope}	
}

%Strechable skip to get contents a the bottom of slides
\newcommand{\btVFill}{\vskip0pt plus 1filll}

\makeatletter
\newcommand*{\defeq}{\mathrel{\rlap{%
			\raisebox{0.3ex}{$\m@th\cdot$}}%
		\raisebox{-0.3ex}{$\m@th\cdot$}}%
	=}
\makeatother

\newcommand{\ddiff}{\mathrm{d}}
\newcommand{\R}{\ensuremath{\mathbb{R}}}
\newcommand{\C}{\ensuremath{\mathbb{C}}}

\newcommand{\bmat}[1]{\ensuremath{\begin{bmatrix} #1 \end{bmatrix}}}

\newcommand{\imgsource}[1]{\begin{flushright}
		\textcolor{black!40}{\tiny Source:~#1}	
\end{flushright}}

\newcommand{\follows}{\ensuremath{\Rightarrow}}
\newcommand{\gdw}{\ensuremath{\Leftrightarrow}}

\usetikzlibrary{backgrounds}

%\usepackage{tikz-uml}
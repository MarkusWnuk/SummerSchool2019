\definecolor{uniSblue}{RGB}{0,65,145}
\definecolor{uniSlightblue}{RGB}{0,190,255}
\definecolor{uniSgrey}{RGB}{62, 68, 76}
\definecolor{uniSyellow}{RGB}{255, 213, 0}

% colors not covered by CD but still useful
\definecolor{uniSred}{RGB}{230, 0, 50}
\definecolor{uniSgreen}{RGB}{0, 200, 50} 

%\usetheme{Luebeck}
\usetheme{boxes}
%\usecolortheme{beaver} %rot
\usecolortheme{seagull} %grau

%\usepackage[utf8]{inputenc}
\usepackage[T1]{fontenc}

\usepackage{fontspec}
\defaultfontfeatures{Ligatures=TeX}
%\usepackage[]{unicode-math} 
%\unimathsetup{math-style=TeX}
%\setmainfont[SmallCapsFeatures={LetterSpace=6}, Numbers={Proportional,OldStyle}, Ligatures=TeX]{Univers for UniS 55 Roman Rg}
%\setmonofont[Scale=0.9]{Hack}
%\setmainfont[SmallCapsFeatures={LetterSpace=6}, Numbers={Proportional,OldStyle}]{Minion Pro}
%\setmainfont[LetterSpace=3, Numbers={Proportional,OldStyle}, Ligatures=TeX]{Myriad Pro}
\defaultfontfeatures{Mapping=tex-text}
%

\usepackage{tikz, pgfplots}

\usepackage[ngerman, english]{babel}% Silbentrennung

%do not ovelap with footer line
\addtobeamertemplate{footnote}{}{\vspace{2.5ex}}

%footnotes without numbering with \blankfootnote{}
\let\svthefootnote\thefootnote
\textheight 1in
\newcommand\blankfootnote[1]{%
	\let\thefootnote\relax\footnotetext{\tiny #1}%
	\let\thefootnote\svthefootnote%
}



%item color setzen
\setbeamercolor{itemize item}{fg=uniSlightblue}
\setbeamercolor{itemize subitem}{fg=uniSlightblue}


\setbeamercolor{block title}{bg=uniSlightblue,fg=white}
\setbeamercolor{block body}{bg=uniSlightblue!10,fg=black}

\setbeamercolor*{structure}{bg=uniSlightblue,fg=uniSlightblue}

\setbeamercolor*{palette primary}{use=structure,fg=uniSlightblue,bg=structure.fg}
\setbeamercolor*{palette secondary}{use=structure,fg=uniSlightblue,bg=structure.fg!75}
\setbeamercolor*{palette tertiary}{use=structure,fg=white,bg=black}
\setbeamercolor*{palette quaternary}{fg=white,bg=black}

\setbeamercolor{section in toc}{fg=black,bg=white}
\setbeamercolor{alerted text}{use=structure,fg=structure.fg!50!black!80!black}

\setbeamercolor{titlelike}{parent=palette primary,fg=structure.fg!50!black}
\setbeamercolor{frametitle}{bg=white,fg=uniSgrey}

\setbeamercolor*{titlelike}{parent=palette primary}


%\setbeamercovered{transparent}

\setbeamertemplate{navigation symbols}{}%remove navigation symbols
\setbeamertemplate{itemize items}[circle]



% http://tex.stackexchange.com/questions/55806/tikzpicture-in-beamer/55827#55827
  \tikzset{
    invisible/.style={opacity=0},
    visible on/.style={alt=#1{}{invisible}},
    alt/.code args={<#1>#2#3}{%
      \alt<#1>{\pgfkeysalso{#2}}{\pgfkeysalso{#3}} % \pgfkeysalso doesn't change the path
    },
  }
  \tikzset{onslide/.code args={<#1>#2}{%
  \only<#1>{\pgfkeysalso{#2}} % \pgfkeysalso doesn't change the path
}}




\newcommand{\partitle}[1]{\textcolor{uniSlightblue}{\textbf{#1}}\\ \smallskip}

% argument #1: any options
\newenvironment{customlegend}[1][]{%
	\begingroup
	% inits/clears the lists (which might be populated from previous
	% axes):
	\csname pgfplots@init@cleared@structures\endcsname
	\pgfplotsset{#1}%
}{%
	% draws the legend:
	\csname pgfplots@createlegend\endcsname
	\endgroup
}%

\pgfplotsset{compat=newest}


% makes \addlegendimage available (typically only available within an
% axis environment):
\def\addlegendimage{\csname pgfplots@addlegendimage\endcsname}
\def\addlegendentry{\csname pgfplots@addlegendentry\endcsname}



